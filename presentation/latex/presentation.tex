% \PassOptionsToPackage{draft}{graphicx}
\documentclass[10pt]{beamer}

\usetheme[]{metropolis}

\usepackage{inputenc}
\usepackage[english]{babel}
\usepackage{tabularx}
\usepackage{booktabs}
\usepackage{gensymb}
\usepackage{ragged2e}
\usepackage{amsmath}
\usepackage{xcolor}
\usepackage{textcomp}
\usepackage{amssymb}
\usepackage{array}
\usepackage{fontawesome}
\usepackage{threeparttable}
\usepackage[font=tiny,labelfont=bf]{caption}
\usepackage[style=authoryear, sorting=nyt, natbib, isbn=false,url=false, doi=false]{biblatex}
\addbibresource{references.bib}

% define tikz environment properties
\usepackage{tikz}
\usepackage{tikzscale}
\usetikzlibrary[external]
\tikzexternaldisable
\usepackage{pgfplots}
\usetikzlibrary{backgrounds,calc,patterns,positioning,shapes,decorations.pathreplacing,decorations.pathmorphing,shapes.geometric}
\def\Plus{\texttt{+}}
\def\Minus{\texttt{-}}
\definecolor{bg-white}{HTML}{FAFAFA}
\definecolor{cgreen}{HTML}{a5d6a7}
\definecolor{corange}{HTML}{ffcc80}
\definecolor{cgreend}{HTML}{4caf50}
\definecolor{coranged}{HTML}{ff9800}
\definecolor{tdf-green}{HTML}{267300}
\definecolor{tdf-orange}{HTML}{FFAA00}


\usepackage{xspace}
\newcommand{\themename}{\textbf{\textsc{metropolis}}\xspace}

\title{Fire activity on the Colombian Tropical Dry Forest}
\institute{Faculty of Environmental and Rural Studies \\ Pontificia Universidad Javeriana \\ Bogotá, Colombia}
\subtitle{An environmental and social perspective}
\date{November 21, 2019}
\author{Marcelo Villa-Piñeros}
% \titlegraphic{\hfill\includegraphics[height=1.5cm]{logo.pdf}}

\begin{document}

\maketitle

\section{Why the Tropical Dry Forest?}

\begin{frame}{}
    \centering
    Highly \alert{threatened} and \alert{understudied} biome
\end{frame}

\section{What is the Tropical Dry Forest?}
{%
\setbeamertemplate{frame footer}{\tiny{\citet{Mooney1995,Sanchez-Azofeifa2005,Dirzo2011,Pizano2014b,Dexter2018}}}
\begin{frame}{}
    \begin{columns}[onlytextwidth]
        \begin{column}{0.55\textwidth}
            \begin{itemize}
                \item Deciduous vegetation
                \item Relatively high fertility
                \item Limited to lowlands (1000 m.s.n.m.)
                \item Annual temperature: > 25\degree C
                \item Annual precipitation: between 250 and 2500 mm
                \item \alert{Strong seasonality}
            \end{itemize}
        \end{column}
        \begin{column}{0.45\textwidth}
            \begin{figure}
                \centering
                \includegraphics[width=\textwidth]{figures/misc/tuparro-bosques-secos}
                \caption{Tropical Dry Forest in the El Tuparro natural national park, Colombia. Credits: Instituto Alexander von Humboldt.}
            \end{figure}
        \end{column}
    \end{columns}
\end{frame}
}

\section{Where is the Tropical Dry Forest?}
\begin{frame}{}
    \begin{figure}
        \centering
        \includegraphics[width=\textwidth]{figures/map/tdf_world}
        \caption{\textcolor{tdf-green}{\textbf{Tropical and Subtropical Dry Broadleaf Forests}} ecoregion. Data sources: \citet{Olson2002} and Natural Earth.}
    \end{figure}
\end{frame}


\begin{frame}{}
    \begin{figure}
        \centering
        \includegraphics[width=\textwidth]{figures/map/tdf_col}
        \caption{Tropical Dry Forest \textcolor{tdf-orange}{\textbf{biome}} and \textcolor{tdf-green}{\textbf{2014 distribution}} in Colombia. Data sources: \citet{Ariza2014,Etter2015}, CGIAR/SRTM, Natural Earth.}
    \end{figure}
\end{frame}

\section{What is the situation of the Tropical Dry Forest?}

\begin{frame}{}
    \vspace*{8mm}

    \onslide<1->{
        \begin{quote}
            ``The rain forest is not the most threatened of the major tropical forest types. The \alert{tropical dry forests} hold this honor.''
        \end{quote} 
        \rightline{\footnotesize{--- \citet{Janzen1988}}}
    }
    
    \vspace{5mm}

    \onslide<2->{
        \begin{quote}
            ``Virtually all of the \alert{tropical dry forests} that remain are currently exposed to a variety of different threats.''
        \end{quote}
        \rightline{\footnotesize{--- \citet{Miles2006}}}
    }
    \vspace{5mm}
    
    \onslide<3>{
        \begin{quote}
            ``There is a general perception that the most urgent conservation issues in the tropics concern rain forests but their disproportionate scientific, policy and public profiles have distracted attention from the vulnerability of \alert{tropical dry forests}.''
        \end{quote}
        \rightline{\footnotesize{--- \citet{Pennington2018}}}
    }
\end{frame}

{%
\setbeamertemplate{frame footer}{\tiny{\citet{Janzen1988,Miles2006,Garcia2014,Etter2015,Dexter2018,Pennington2018}}}
\begin{frame}{}
    \begin{columns}[onlytextwidth]
        \begin{column}{0.45\textwidth}
            \textbf{Threats}
            \begin{itemize}
                \item Climate change
                \item Forest fragmentation
                \item Conversion to agriculture
                \item Human population
                \item \alert{Fire}
            \end{itemize}
        \end{column}
        \begin{column}{0.55\textwidth}
            \textbf{Status}
            \begin{itemize}
                \item 3\% unexposed to threats globally
                \item 10\% remains intact in Latin America
                \item \alert{3\% remaining in Colombia (critically endangered)}
            \end{itemize}
        \end{column}
    \end{columns}
\end{frame}
}

\section{Fire activity relationships in Tropical Forests}

\begin{frame}{}
    \begin{figure}
        \centering
        \includegraphics[width=\textwidth]{figures/tikz/controls_and_pressures.tikz}
        \caption{Theoretical model of environmental controls and human pressures regulating fire activity in tropical forests. Based on \citet{Cochrane2003,Shlisky2009,Bowman2011}}
    \end{figure}
\end{frame}

\section{Data}

\begin{frame}{}
    \begin{table}
    \scriptsize
    \caption{Retrieved datasets and their properties}
    \centering
    \begin{tabular}{lllll}
        \toprule
        Dataset & Source & Product & Temporal res.  & Spatial res.  \\
        \midrule            
        Thermal Anomalies & MODIS & MOD14A2 & Weekly & 1 km \\
        Precipitation & TRMM & 3B43 & Monthly & 0.25\degree \\
        Enhanced Vegetation Index  & MODIS & MOD14A3 & Monthly & 1 km \\
        Land cover & MODIS & MCD12Q1 & Yearly  & 500 m \\
        \bottomrule
    \end{tabular}
    \end{table}
\end{frame}

\begin{frame}{}
    \begin{figure}
        \centering
        \includegraphics{figures/tikz/data_model.tikz}
        \caption{Data model representation}
    \end{figure}
\end{frame}

\section{Environmental controls overview}
\begin{frame}{}
    \begin{figure}
        \centering
        \includegraphics[width=0.9\textwidth]{figures/graph/fire_ppt_evi_time_series}
        \caption{Fire pixels, Precipitation and Enhanced Vegetation Index time series (left side) and month aggregation (right side). Red line represents the dry month precipitation threshold \citep{VanDerWerf2008}.}
    \end{figure}
\end{frame}

\begin{frame}{}
    \begin{figure}
        \centering
        \includegraphics[width=0.9\textwidth]{figures/graph/ppt_evi_kde}
        \caption{Precipitation and Enhanced Vegetation Index \textbf{all-pixel values distribution} and \alert{\textbf{fire-pixel-only values distribution}}.}
    \end{figure}
\end{frame}

\begin{frame}{}
    \begin{figure}
        \centering
        \includegraphics[width=0.9\textwidth]{figures/graph/ppt_evi_correlation.eps}
        \caption{Correlation between fire pixels and both Precipitation and Enhanced Vegetation Index (including previous 3 month period average). A lowess model is fitted and the Pearson correlation coefficient is displayed for each variable.}
    \end{figure}
\end{frame}

\section{Human pressures overview}

\begin{frame}{}
    \begin{columns}[onlytextwidth]
        \begin{column}{0.5\textwidth}
            \begin{figure}
                \centering
                 \includegraphics[width=0.95\textwidth]{figures/graph/landcover_treemap}
                \caption{Proportion of land cover type.}
            \end{figure}
        \end{column}
        \begin{column}{0.5\textwidth}
            \begin{figure}
                \centering
                 \includegraphics[width=0.95\textwidth]{figures/graph/fire_per_landcover_boxplot}
                \caption{Fire pixel proportion per land cover type.}
            \end{figure}
        \end{column}
    \end{columns}
\end{frame}

\begin{frame}{}
    \begin{figure}
        \centering
        \includegraphics[width=0.9\textwidth]{figures/graph/distance_to_nearest_forest_hist}
        \caption{Distance to the nearest forest pixel values distribution. Each bar corresponds to distance of one pixel. Red bar represents pixels with zero distance (\emph{i.e.} forest pixels).}
    \end{figure}
\end{frame}

\begin{frame}{}
    \begin{figure}
        \centering
        \includegraphics[width=1\textwidth]{figures/graph/distance_by_landcover_reg}
        \caption{Fire pixel probability as a function of distance to the nearest forest pixel and land cover type.}
    \end{figure}
\end{frame}

\section{Further thoughts}

\section{The \emph{Cerro Los Cristales} Fire}

\begin{frame}{}
    \begin{columns}[onlytextwidth]
        \begin{column}{0.5\textwidth}
            \begin{figure}
                \centering
                \includegraphics[width=0.95\textwidth]{figures/misc/cerro-los-cirstales-fire-map}
                \caption{Burned area vs. restoration project area. Source: Grupo Conservación de Ecosistemas.}
            \end{figure}
        \end{column}
         \begin{column}{0.5\textwidth}
            \begin{figure}
                \centering
                 \includegraphics[width=0.95\textwidth]{figures/misc/cerro-los-cristales-fire}
                \caption{Fire on the \emph{Cerro Los Cristales}, 31 August 2018. Credits: Grupo Conservación de Ecosistemas.}
            \end{figure}
        \end{column}
    \end{columns}
\end{frame}

\section{Conclusions}

\begin{frame}{}
    \begin{itemize}[<+->]
        \item \alert{Climate} as a dominant control
        \item \alert{Human shaped} burning patterns
        \item Important underlying \alert{social} and \alert{economic} factors
    \end{itemize}
\end{frame}

\begin{frame}{}
    \centering
    \faGithub~~github.com/\alert{marcelo-villa-p}/fire-congress-2019-tdf-talk
\end{frame}


\appendix

\begin{frame}[allowframebreaks]{}
    \renewcommand*{\bibfont}{\tiny}
    \printbibliography[heading=none]
%   \bibliographystyle{abbrv}

\end{frame}


\end{document}
